\documentclass[a4paper,11pt]{scrartcl}

% Formatting
\usepackage[a4paper,
inner=2.5cm, 
outer=2.5cm,
top=3.0cm,
bottom=3.0cm]{geometry}
\usepackage{indentfirst}


% Language settings
\usepackage{polyglossia}
\setdefaultlanguage{english}


% Packages
\usepackage{mathtools}
\usepackage{amsmath}
\usepackage[standard,thmmarks,thref,amsmath,hyperref]{ntheorem}
\usepackage{braket}
\usepackage{xcolor}
\usepackage{bm}

\usepackage{hyperref}
\hypersetup{
	colorlinks,
	linkcolor={red!50!black},
	citecolor={blue!50!black},
	urlcolor={blue!80!black}
}

\usepackage[%
	backend=biber,
	sorting=nyt
	]{biblatex}
\addbibresource{docs.bib}

% Commands
\newcommand{\F}[1][R]{\mathbb{#1}}

\definecolor{darkgreen}{rgb}{0.0,0.5,0.0}

\newcommand{\rt}[1]{\textcolor{red}{#1}}
\newcommand{\gt}[1]{\textcolor{darkgreen}{#1}}
\newcommand{\bt}[1]{\textcolor{blue}{#1}}

\newcommand{\Ltwonorm}[1]{\left\Vert #1 \right\Vert}
\newcommand{\Eucprod}[2]{\left\langle #1,#2 \right\rangle}
\newcommand{\rp}{\mathop{\mathfrak{Re}}}
\newcommand{\ip}{\mathop{\mathfrak{Im}}}
\newcommand{\setsep}{\ \middle|\ }


% Math environments
\newcounter{cnt}

\theoremsymbol{\scriptsize$ \square $}
\theoremseparator{.}
\theorembodyfont{}
\newtheorem{proposition1}[cnt]{Proposition}

\theoremsymbol{\scriptsize$ \square $}
\theoremseparator{.}
\theorembodyfont{}
\newtheorem{definition1}[cnt]{Definition}

\theoremsymbol{\scriptsize$ \square $}
\theoremseparator{.}
\theorembodyfont{}
\newtheorem{remark1}[cnt]{Remark}

\theoremsymbol{\scriptsize$ \square $}
\theoremseparator{.}
\theorembodyfont{}
\newtheorem{conclusion1}[cnt]{Conclusion}

\theoremsymbol{\scriptsize$ \square $}
\theoremseparator{.}
\theorembodyfont{}
\newtheorem{lemma1}[cnt]{Lemma}



% Title page
\title{Fast Spherical Harmonics Transform}
\subtitle{Notes on Implementation (rough draft)}
\date{\today}
\author{Ido Hadi}

\begin{document}
	\maketitle
	
	\tableofcontents
	
	\section{Conventions and Notations}
	
	\subsection{Polar Coordinates on $ \F^{2} $}
	Every point $ (x,y) \in \F^{2} $ is represented in \textbf{polar coordinates} as $ x = \rho \cos \theta $ and $ y = \rho \sin \theta $ where $ \rho \ge 0 $ and $ \theta \in [0, 2 \pi) $. $ (0,0) $ is usually represented by $ \rho = 0 $ and $ \theta = 0 $.
	
	\subsection{Spherical Coordinates on $ \F^{3} $}
	Every point $ (x,y,z) \in \F^{3} $ is represented in \textbf{spherical coordinates} as $ x = \rho \cos \theta \sin \phi $, $ y = \rho \sin \theta \sin \phi $ and $ z = \cos \phi  $ where $ \rho \ge 0 $, $ \theta \in [0, 2 \pi) $ and $ \phi \in [0, \pi ) $. $ (0,0,0) $ is usually represented by $ \rho = 0 $, $ \theta = 0 $ and $ \phi = 0 $.
	
	\subsection{Geometry}
	
	\begin{definition1}
		Let $ n > 0 $ be an integer. The \textbf{$ \bm{p} $-norm} on $ \F^{n} $ or $ \F[C]^{n} $ is
		\begin{equation*}
		\Ltwonorm{\mathbf{x}}_{p}
		= \left( \sum_{j=1}^{n} \left|x_i\right|^{p} \right)^{1/p}
		\end{equation*}
	\end{definition1}
	
	\begin{definition1}
		The \textbf{$ \bm{n} $-sphere} is $ S^2 = \left\{ \mathbf{x} \in \F^{n} \setsep \Ltwonorm{\mathbf{x}}_2 = 1 \right\} $. The $ 2 $-sphere is referred to simply as the sphere.
	\end{definition1}
	
	\section{TODO}
	
	\printbibliography
	
\end{document}